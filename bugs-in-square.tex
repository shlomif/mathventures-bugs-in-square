\documentclass[a4paper]{article}

\usepackage[]{fullpage}

\usepackage[]{amsmath}

% This statement puts a one-line spacing between two adjacent paragraphs
\setlength\parskip{\medskipamount}

% This statement cancels the indentation of the paragraph's first line
\setlength\parindent{0pt}

\begin{document}

I'll use the following model to simplify the problem: suppose the bugs don't
move towards each other infinitesimally, but rather move in a straight line, and
then stop, and then move again toward their new positions, and so forth. The
bugs move toward their initial position until they reach a position which is in
proportion p to the initial distance between them. Then they move again,
towards a position which is in proportion p to their distance, and so forth.

Thus they form an infinite series of squares inside each other. You can see an
illustration of this scheme to the right of this text for the proportion
$ p=0.15 $.

Now, if the length of a given square is $a$, then the length of the square
inside it is (according to Pythagoras theorem):

\begin{eqnarray*}
& & \sqrt{\left(p*a\right)^2+\left(\left(1-p\right)*a\right)^2} =
a\sqrt{2p^2-2p+1}
\end{eqnarray*}
The lengths of the squares form a decreasing
geometrical series, with that proportion. Thus the length of the path a bug
travel until they meet is:

\begin{eqnarray*}
& & \frac{p \cdot a_1}{1 - \sqrt{2p^2-2p+1} }
\end{eqnarray*}

This is according to the formula that the sum of an infinite decreasing
geometric series is $\dfrac{a_1}{1-q}$ where $a_1$ is the value of its first item
and $q$ is the proportion between two consecutive items.

Now, to find the length an infinitesimal bug will travel, we just limit $p$ to
0:

\begin{eqnarray*}
& & \lim\limits_{p \to 0} \frac{p \cdot a_1}{1-\sqrt{2p^2-2p+1}} = \\
& & \lim\limits_{p \to 0} \frac{p \cdot a_1}{1-\sqrt{2p^2-2p+1}} \cdot
\frac{1+\sqrt{2p^2-2p+1}}{1+\sqrt{2p^2-2p+1}} = \\
& & \lim\limits_{p \to 0} \frac{a_1 \cdot p \cdot \left(1+\sqrt{2p^2-2p+1}\right)}{1-p^2+2p-1} = \\
& & \lim\limits_{p \to 0} \frac{a_1 \cdot \left(1+\sqrt{2p^2-2p+1}\right)}{2-2p} = \\
& & \lim\limits_{p \to 0} \frac{a_1 \cdot \left(1+\sqrt{2 \cdot 0^2-2 \cdot 0+1}\right)}{2-2 \cdot 0} = \\
& & \frac{a_1 \cdot (1+1)}{2} = a_1
\end{eqnarray*}

Therefore, the length of a bug's path is equal to the length of the original
square's side. The time it will take a 1 meter per second fast bugs who stand
at the corner of a 1*1 meter square to meet is 1/1 = 1 second.

\end{document}

